\documentclass[conference]{IEEEtran}
\IEEEoverridecommandlockouts

\usepackage{textcomp}
\usepackage{graphicx,cite,epsfig,amssymb,amsmath,amsfonts,multicol,subfigure,mathtools,bm,mathrsfs,setspace,algorithm,algorithmic,amsthm,dblfloatfix, xcolor, booktabs, enumitem}

\usepackage{caption}
\usepackage[UTF8]{ctex} 

\usepackage{lettrine}
\allowdisplaybreaks[4]
\hyphenation{op-tical net-works semi-conduc-tor}
\theoremstyle{plain}
\newtheorem{lemma}{Lemma}

\begin{document}
	\title{基于知识库不对称性的安全语义广播系统}
	
	\author{杨壮旭
	}
	
	\maketitle


\begin{abstract}


\end{abstract} 

\begin{IEEEkeywords}
语义通信,广播系统,背景知识库,语义安全
\end{IEEEkeywords}


\section{Introduction}

\section{system model}
本节给出所研究的安全语义广播系统的整体模型与信号发送流程,重点说明系统架构、知识库构建以及接收端可选择性的语义解码方法。

\subsection{系统架构}
系统采用单发送端、多接收端的广播结构。在本文中,我们称合法发送端为 Alice,称三个合法接收端为 Bob$_1$、Bob$_2$ 与 Bob$_3$,物理层采用的是共享广播结构。广播结构的一次传输过程时,发送端Alice 仅发送一份公共信号,所有接收端( Bob$_1$、Bob$_2$、Bob$_3$)均能接收到该信号。与传统仅依赖物理层保密的方案不同,本系统在语义层面引入了基于知识库不对称性的机制,实现了对广播消息的选择性语义可解码:在一次广播过程中,仅预定的目标接收端能够正确恢复源语义信息,其余接收端虽可完成解码,但在语义层面得到的解码结果是错误或无意义的,从而实现语义的可控解码。例如:这次发送过程中合法发送端Alice想把信息发送给Bob$_1$,Bob$_1$、Bob$_2$ 与 Bob$_3$都能接收到Alice发出的信号,但是只有Bob$_1$能够正确恢复出Alice想发送的语义信息,Bob$_2$和Bob$_3$解码出来的语义信息是不正确的。具体系统架构可见 Fig. \ref{system}      

系统中使用了两种典型的无线信道模型:加性高斯白噪声(AWGN)信道与瑞丽衰落信道。在测试时分别使用了这两种信道来评估系统的性能:

\subsubsection{AWGN 信道模型}

系统首先采用 AWGN 信道进行建模。该信道假设发送信号仅受到独立同分布的高斯白噪声影响,不包含任何衰落、多径等变化因素。其接收端信号可表示为
\begin{equation}
	y = x + n,
\end{equation}
其中 $x$ 为 Alice 发送的信道符号, $y$ 为接收端Bob所接受到的信道符号,$n \sim \mathcal{N}(0,\sigma^2)$ 为零均值且方差为 $\sigma^2$ 的加性高斯白噪声。

\subsubsection{瑞丽衰落信道模型}

为了评估系统在更加真实的无线传播环境中的表现,本文还采用了瑞丽衰落信道进行建模。其接收端信号可表示为
\begin{equation}
	y = h x + n,
\end{equation}
其中 $x$ 为 Alice 发送的信道符号, $y$ 为接收端Bob所接受到的信道符号,$h \sim \mathcal{CN}(0,1)$ 为信道增益,这是服从复高斯分布的随机变量,其幅度服从瑞利分布,$n$ 为加性高斯白噪声。

由于系统采用广播传输方式,Alice 的信道符号 $x$ 在同一时间被同时发送给所有接收端。三个接收端 Bob$_1$、Bob$_2$、Bob$_3$ 在物理层的接收信号可统一表示为
\begin{equation}
	y_i = h_i x + n_i, \quad i \in \{1,2,3\},
\end{equation}
其中 $h_i$ 和 $n_i$ 分别表示 Alice 到 Bob$_i$ 的信道增益与噪声。  
在 AWGN 信道下, $h_i = 1$;在瑞丽信道下,$h_i$ 为独立且同分布的瑞利衰落系数。由于广播信道不为不同用户分配独立资源,因此所有接收端Bob$_i$ 在物理层均可获得一致结构的信号,仅在信道增益和噪声上有区别。

\begin{figure*}[t]
	\centering
	\includegraphics[width=\textwidth]{system.png}
	\caption{system}
	\label{system}
\end{figure*}

 

\subsection{知识库构建}						
系统中存在一份全局背景知识集合$\mathcal{K}$,表示语义通信的先验信息。Alice 从 $\mathcal{K}$ 中提取与当前任务相关的部分,构建标准知识库 $K$,并在此基础上进行语义编码。

三个接收端 Bob$_1$、Bob$_2$、Bob$_3$ 分别持有本地知识库 $K_1$、$K_2$、$K_3$。需要说明的是:
\begin{itemize}
	\item 在内容层面,$K_1$、$K_2$、$K_3$ 所包含的知识条目完全相同,均来源于同一背景知识集合 $\mathcal{K}$;
	\item 在索引层面,各接收端在构建本地知识库时采用了不同的知识索引顺序。可表示为对标准知识库 $K$ 使用不同的排列映射 $\pi_i$:
	\begin{equation}
		K_i = \pi_i(K), \quad i \in \{1,2,3\},
	\end{equation}
	 Bob$_1$ ,Bob$_2$ 与 Bob$_3$ 分别采用不同的排列$\pi_1,\pi_2,\pi_3$。
\end{itemize}

上述内容相同但索引不同的构造方式,使得在向量空间中,同一维度在不同接收端上对应到的知识库发生错位,即各接收端在语义空间中知识的对应关系不同。并且由于所有接收端的知识库都是从标准知识库构建而来的,因此对于发送端Alice来说,它知道所有接收端的知识库信息。这一结构为实现仅目标接收端可正确语义解码奠定了基础。

\subsection{编码和解码过程}
\subsubsection{编码过程}
发送端 Alice 的编码过程由语义编码与信道编码两部分组成,其目标是将原信息映射为语义信息,再将语义信息映射为可在无线信道中传输的符号序列。

以文本信息传输为例,向系统中输入一个句子,$s = [w_1,w_2,\dots,w_l]$,其中 $w_l$ 代表句子中的第 $l$ 个单词。首先,Alice要确定本次要传输的目标接收端 Bob$_i$,从标准知识库中提取出对应的本地知识库$K_i$,结合输入的句子 $s$,通过语义编码器生成语义嵌入向量
\begin{equation}
	r = f_{\mathrm{S}}(s, K_i),
\end{equation}
其中 $f_{\mathrm{S}}(\cdot)$ 表示语义编码网络。语义向量 $s$ 的每个维度与知识库中的索引严格对应,从而保持稳定的语义结构。

随后,语义表示 $s$ 被送入信道编码器,映射为物理层可传输的符号序列
\begin{equation}
	x = f_{\mathrm{C}}(r),
\end{equation}
其中 $f_{\mathrm{C}}(\cdot)$ 表示信道编码模块。语义编码层以 Transformer 结构为核心,使用多个 Transformer 编码器来提取输入信息的语义特征;信道编码器则使用了不同输出单元的全连接层。同时为了防止其中的关键语义信息被干扰,采用自编码器结构来搭建网络。




\begin{thebibliography}{10}
	
	
	
	\bibitem{ref1}
	H. Xie, Z. Qin, G. Y. Li and B. -H. Juang, "Deep Learning Enabled Semantic Communication Systems," in IEEE Transactions on Signal Processing, vol. 69, pp. 2663-2675, 2021, doi: 10.1109/TSP.2021.3071210.
	
	
\end{thebibliography}
\end{document}

